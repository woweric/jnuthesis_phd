% !TeX spellcheck = en_GB
% !Mode:: "TeX:UTF-8"
% !TeX encoding = UTF-8 Unicode
% !TeX program = xelatex
% !TeX root = ../root.tex


\chapter{论文模板使用说明}
\label{ch99}


\section{模板介绍}

为了方便撰写江南大学学位论文, 设计本 \LaTeX 模板.
本模板采用 MIT 协议授权, 如有不完善之处, 请自行修改代码.

为了在macOS Sonoma 14.4.1上运行,我们对原作者Bo Zhuang(bozhuang@jiangnan.edu.cn)的部分代码进行修改,主要包括:

1. 替换macOS系统字体

2. 增加算法代码样式

修改日期: 2024/5/7

联系邮箱: mr.qzhu@gmail.com

\subsubsection*{主要内容}

本模板主要包括两个文件:

\begin{tabular}{ll}
\hline
文件 & 说明\\
\hline
{\tt{jnthesis.cls}} & 提供文档类 {\tt{jnthesis}}, 包含论文各部分的格式设置\\
{\tt{jn.bst}} & 提供参考文献格式\\
\hline
\end{tabular}

\subsubsection*{文档类 {\tt{jnthesis.cls}}}

文档类 {\tt jnthesis} 基于 {\tt ctex} 宏包提供的 {\tt ctexbook} 文档类,
依据江南大学学位论文格式(2012版)进行排版,
定义了字体, 字号, 行距, 标题, 页眉, 页脚, 目录, 摘要, 正文等各种格式.
基本用法如下:

\begin{lstlisting}[basicstyle=\ttfamily, frame=single]
\documentclass{jnthesis}
\end{lstlisting}

上述文档默认为博士学位论文, 可以通过设置参数 {\tt doctor} 或 {\tt master} 指定为博士或硕士学位论文, 还可以指定参数 {\tt nodegree} 表示毕业论文而非学位论文,
用法如下:
\begin{lstlisting}[basicstyle=\ttfamily, frame=single]
\documentclass[doctor]{jnthesis} % 博士学位论文
\documentclass[master]{jnthesis} % 硕士学位论文
\documentclass[doctor, nodegree]{jnthesis} % 博士毕业论文
\end{lstlisting}

文档类 {\tt jnthesis} 针对论文结构, 提供了相应的命令.

\begin{lstlisting}[basicstyle=\ttfamily, frame=single]
    \jnfrontmatter                     % 论文开始
    \jnabstract[zh]                    % 中文摘要
    \begin{jnkeywords}
      学位论文 \sep 江南大学 \sep 博士 \sep 硕士
    \end{jnkeywords}
    \jnabstract[en]                    % 英文摘要
    \begin{jnkeywords}[en]
      Thesis \sep Jiangnan University \sep Doctor \sep Master
    \end{jnkeywords}
    \jncontents                        % 目录
    \jnmainmatter                      % 正文开始
    \jnacknowledgements                % 致谢
    \jnreferences                      % 参考文献
    \jnpublications                    % 发表论文列表
\end{lstlisting}

文档类 {\tt jnthesis} 还定义了以下命令:

\begin{lstlisting}[basicstyle=\ttfamily, frame=single]
    \sep                    % 关键词之间的分隔符
    \setbstfile{jn}         % 设置 bst 文件
    \setbibfiles{ref1,ref2} % 指定 bib 文献数据库
    \citeit{refkey}         % 在正文中引用参考文献
\end{lstlisting}

文档类 {\tt jnthesis} 还定义了定理环境:

\begin{lstlisting}[basicstyle=\ttfamily, frame=single]
\begin{theorem}定理\end{theorem}
\end{lstlisting}

类似的还有 {\tt lemma} 引理,
{\tt proposition} 命题,
{\tt assumption} 假设,
{\tt corollary} 推论,
{\tt property} 性质,
{\tt axiom} 公理,
{\tt definition} 定义,
{\tt example} 例,
{\tt remark} 注.


文档类 {\tt jnthesis} 导入了下列宏包,
其具体用法请查阅相关文档.

\begin{tabular}{lll}
\hline
功能 & 宏包 & 参数\\
\hline
标题, 目录 & titlesec, titletoc &\\
页眉页脚 & fancyhdr &\\
字体, 行距 & fontspec, xunicode, setspace &\\
列表 & enumerate, enumitem\\
表格 & booktabs, longtable, &\\
& hhline, threeparttable & \\
参考文献 & natbib & square, super, comma, sort\&compress\\
\hline
数学公式 & amsmath & \\
数学字体 & amsfonts, mathrsfs, mathtools & \\
定理环境 & ntheorem & hyperref, thmmarks, amsmath\\
算法, 代码 & algorithmicx, algpseudocode, listings & \\
\hline
插图, 子图 & graphicx, subfig &\\
双语标题 & bicaption &\\
颜色, 超链接 & color, hyperref &\\
\hline
\end{tabular}


\subsubsection*{参考文献格式 {\tt{jn.bst}}}

参考文献格式 {\tt jn} 可以配合 BibTeX 文献数据库 (.bib) 实现参考文献格式化.

\begin{lstlisting}[basicstyle=\ttfamily, frame=single]
\bibliographystyle{jn}  % 指定参考文献格式
\end{lstlisting}

参考文献具体格式参见文后的参考文献部分\cite{Dupont1974,Aho1986,Chen1990,张全福1991,高景德1987,余勇1998,霍夫斯基1981,张竹生1983,竺可桢1973}.

采用 {\tt natbib} 格式化参考文献的引用.
比如可以用 {\tt cite} 命令引用参考文献,
如\cite{Dupont1974,Aho1986,Chen1990,竺可桢1973},
同时定义了命令 {\tt citeit} 用于在文章内容中引用具体文献,
如文献\citeit{Dupont1974,Aho1986,Chen1990,竺可桢1973}.

对于专业用户, 如果个别类型的文献格式不符合要求, 请修改格式文件 {\tt jn.bst}.

对于一般用户, 如果个别类型的文献格式不符合要求, 请直接修改编译生成的 {\tt root.bbl} 文件.

\subparagraph{注意:} 为生成参考文献, 需执行 {\tt BibTeX} 命令. 该命令生成并重写 {\tt root.bbl} 文件.

\subparagraph{编写 {\tt .bib} 文献数据库常见的问题:}
(1) 建议利用相关软件编写 BibTeX 文献数据库, 如 JabRef.
(2) 标题中个别字母大写时, 用大括号括起来, 如 {\tt title=\{Boundary control of \{PDEs\}\}}. (3) 注意英文人名的正确写法, 如张三的英文名正确写法是 {\tt San Zhang} 或 {\tt S. Zhang} 或 {\tt Zhang, San} 或 {\tt Zhang, S.} 均可.

\section{模板用法}

\subsection{基本用法}

为便于编辑, 通常将长文档分成若干文件.
这里提供了一个具体的模板,
除了以上所说的 {\tt jnthesis.cls} 和 {\tt jn.bst} 两个文件之外,
还包括以下文件:

\begin{tabular}{ll}
\hline
文件 & 说明\\
\hline
{\tt root.tex} & 主文档, 整个文档结构, 用 XeLaTeX 编译此文档\\
{\tt main.tex} & 主要内容, 包含主要章节内容\\
\hline
{\tt cover.doc} & 封面, DOC 文件, 修改编辑后另存为 PDF\\
{\tt cover.pdf} & 封面, PDF 文件, 插入文档\\
{\tt statement.doc} & 声明和授权, DOC 文件, 修改编辑后另存为 PDF\\
{\tt statement.pdf} & 声明和授权, PDF 文件, 插入文档\\
\hline
{\tt setup/settings.tex} & 用户设置, 如: 标题, 作者, 其他宏包和样式等\\
{\tt setup/userdefs.tex} & 用户自定义符号\\
\hline
{\tt preface/e\_abstract.tex} & 英文摘要和英文关键词\\
{\tt preface/c\_abstract.tex} & 中文摘要和中文关键词\\
\hline
{\tt body/ch01.tex} & 第一章内容\\
{\tt body/ch02.tex} & 第二章内容\\
...... & 各章节内容, 不需要的部分可在 {\tt main.tex} 中删除\\
{\tt body/ch99.tex} & 模板使用说明, 不需要时在 {\tt main.tex} 中删除\\
\hline
{\tt appendix/acknowledgements.tex} & 致谢内容\\
{\tt appendix/publications.tex} & 发表论文\\
\hline
{\tt references.bib} & 参考文献数据库\\
{\tt figures/...} & 插图\\
\hline
\end{tabular}


具体用法如下:
\subparagraph{1.} 打开 {\tt root.tex} 文件, 设置文档参数以指定博士(doctor), 硕士(master) 或毕业(nodegree)论文.
\subparagraph{2.} 打开 {\tt main.tex} 文件, 规划论文主要章节. 不需要的章节可以删除或注释掉.
\subparagraph{3.} 修改 {\tt cover.doc} 文件生成封面 {\tt cover.pdf}.
\subparagraph{4.} 修改(如有必要) {\tt statement.doc} 生成 {\tt statement.pdf}.
\subparagraph{5.} 修改 {\tt setup/settings.tex} 设置标题, 作者, 包含其他宏包等其他设置.
\subparagraph{6.} 修改(如有必要) {\tt setup/userdefs.tex} 添加用户自定义符号或命令.
\subparagraph{7.} 修改 {\tt preface/e\_abstract.tex} 添加英文摘要和英文关键词.
\subparagraph{8.} 修改 {\tt preface/c\_abstract.tex} 添加中文摘要和中文关键词.
\subparagraph{9.} 修改 {\tt body/ch01.tex} 等, 撰写各章内容.
\subparagraph{10.} 修改 {\tt appendix/acknowledgements.tex} 添加致谢内容.
\subparagraph{11.} 修改 {\tt appendix/publications.tex} 添加发表论文.
\subparagraph{12.} 修改任何内容后, 用 XeLaTeX 编译 {\tt root.tex} 文件得到最终论文 {\tt root.pdf}. 若参考文献不正确, 首先执行 BibTeX, 再多次(三次以上)执行 XeLaTeX, 直到得到正确的参考文献.

一次完整的编译过程为 XeLaTeX > BibTeX > XeLaTeX > XeLaTeX > XeLaTeX. 通常在引用参考文献没有发生变化时, 仅需要执行一次 XeLaTeX.
由于已经把整个文档划分成多个文件, 加快了编译速度.
当 {\tt bib} 文献数据库或文献引用发生变化时, 为确保最终内容正确, 可以执行一遍完整的编译过程.

\subsection{插图}

\subsubsection{设置插图文件路径}

\begin{figure}
	\centering
	\includegraphics{logo/logo-lg.jpg}
	\bicaption{插图例子}{Example of figure}
	\label{fig1}
\end{figure}

\begin{figure}
	\centering
	\subfloat[][子图1]{
		\includegraphics{logo/logo-lg.jpg}}
	\hspace{2em}
	\subfloat[][子图2]{
		\includegraphics{logo/logo-lg.jpg}}
	\caption{子图的例子}
	\label{fig2}
\end{figure}
为方便管理, 建议设置插图文件路径.
例如, 若将插图文件全部存放在 {\tt figures} 文件夹下,
可以设置如下:

\begin{lstlisting}[basicstyle=\ttfamily, frame=single]
\graphicspath{{figures/}} % 插图文件路径 (以 / 结尾)
\end{lstlisting}

此后, 使用 {\tt includegraphics} 命令时, 将在指定的文件夹中搜索.
例如, 为了插入图形文件 {\tt figures/logo/logo-lg.jpg},
可以使用以下命令:

\begin{lstlisting}[basicstyle=\ttfamily, frame=single]
\begin{figure}
  \centering
  \includegraphics{logo/logo-lg.jpg}
  \bicaption{插图例子}{Example of figure}
\end{figure}
\end{lstlisting}

注意其中插图文件名中, 省略了指定的插图路径 {\tt figures/}.
{\tt bicaption} 命令展示了双语标题的用法.
简单标题直接使用 {\tt caption} 命令.
插图效果如图 \ref{fig1} 所示.



\subsubsection{子图}

使用 {\tt subfloat} 命令插入子图(如图 \ref{fig2}).

\begin{lstlisting}[basicstyle=\ttfamily, frame=single]
\begin{figure}
    \centering
    \subfloat[][子图1]{
        \includegraphics{logo/logo-lg.jpg}}
    \hspace{2em}
    \subfloat[][子图2]{
        \includegraphics{logo/logo-lg.jpg}}
    \caption{子图的例子}
    \label{fig2}
\end{figure}
\end{lstlisting}



\section{算法}
如果使用旧模版,请务必在jnthesis.cls文件中加入:
\begin{lstlisting}[basicstyle=\ttfamily, frame=single]
	\RequirePackage{float} % 用来把改中文名
\end{lstlisting}
并增加下列内容:
\begin{lstlisting}[basicstyle=\ttfamily, frame=single]
	\RequirePackage{algorithm}
	\renewcommand{\algorithmicrequire}{ \textbf{输入:}}
	\renewcommand{\algorithmicensure}{ \textbf{输出:}}
	\floatname{algorithm}{算法}
\end{lstlisting}
使用algorithm环境插入算法伪代码:
\begin{lstlisting}[basicstyle=\ttfamily, frame=single]
\begin{algorithm}
	\caption{算法1}
	\label{alg1}
	\begin{algorithmic}
		\Require 这是输入
		\Ensure 这是输出
		\While {flag}
		\State 这是语句
		\EndWhile
	\end{algorithmic}
\end{algorithm}
\end{lstlisting}
效果如下:
\begin{algorithm}
	\caption{算法1}
	\label{alg1}
	\begin{algorithmic}
		\Require 这是输入
		\Ensure 这是输出
		\While {flag}
		\State 这是语句
		\EndWhile
	\end{algorithmic}
\end{algorithm}